\documentclass[10pt,twocolumn]{extarticle}
\usepackage[utf8]{inputenc}
\usepackage[T1]{fontenc}
\usepackage[ngerman]{babel}
\usepackage[scale=0.95]{geometry}
% \usepackage{physics}

\usepackage{amsmath}

% \usepackage[math-style=ISO, bold-style=ISO]{unicode-math}
% \usepackage{fontspec}
% \setmainfont{EB Garamond}
% \setmathfont{Garamond-Math.otf}


\title{Technische Dokument}
\author{Zhongzhu Deng}
\date{\today}

\begin{document}
\maketitle
\tableofcontents



\begin{abstract}
Dies ist eine kurze Zusammenfassung der Inhalte des in deutscher Sprache
verfassten Dokuments.
\end{abstract}

\section{Einleitendes Kapitel}
Dies ist der erste Abschnitt. Hier können wir einige zusätzliche Elemente
hinzufügen und alles wird korrekt geschrieben und umgebrochen werden. Falls ein
Wort für eine Zeile zu lang ist, wird \texttt{babel} versuchen je nach Sprache
richtig zu trennen.

\section{Eingabe mit mathematischer Notation}
In diesem Abschnitt ist zu sehen, was mit Macros, die definiert worden,
geschieht. "` "'

\[ \lim x =  \theta + 152383.5 \]

 
\begin{abstract}
Dies ist eine kurze Zusammenfassung der Inhalte des in deutscher Sprache
verfassten Dokuments.
\end{abstract}

\section{Einleitendes Kapitel}
Dies ist der erste Abschnitt. Hier können wir einige zusätzliche Elemente
hinzufügen und alles wird korrekt geschrieben und umgebrochen werden. Falls ein
Wort für eine Zeile zu lang ist, wird \texttt{babel} versuchen je nach Sprache
richtig zu trennen.

\section{Eingabe mit mathematischer Notation}
In diesem Abschnitt ist zu sehen, was mit Macros, die definiert worden,
geschieht. "` "'

\end{document}